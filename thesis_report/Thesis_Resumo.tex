% !TeX spellcheck = pt_PT
%%%%%%%%%%%%%%%%%%%%%%%%%%%%%%%%%%%%%%%%%%%%%%%%%%%%%%%%%%%%%%%%%%%%%%%%
%                                                                      %
%     File: Thesis_Resumo.tex                                          %
%     Tex Master: Thesis.tex                                           %
%                                                                      %
%     Author: João C. Godinho                                          %
%     Last modified :  May 2018                                        %
%                                                                      %
%%%%%%%%%%%%%%%%%%%%%%%%%%%%%%%%%%%%%%%%%%%%%%%%%%%%%%%%%%%%%%%%%%%%%%%%

\section*{Resumo}

% Add entry in the table of contents as section
\addcontentsline{toc}{section}{Resumo}

O uso de técnicas de aprendizagem supervisionada para a detecção de programas maliciosos tem sido cada vez mais utilizada para melhorar métodos clássicos de detecção.
Este trabalho desenvolveu um modelo para a detecção de programas maliciosos, analisou o impacto da confiabilidade dos dados de treino no resultado final do classificador, e definiu métricas para \textit{verdade absoluta}.
Para tal, desenvolvemos três cenários para conjuntos de dados, cujo conteúdo varia entre amostras inequivocamente maliciosas e legítimas para amostras mais ambíguas e reais.
Analisámos cada cenário em condições laboratoriais, onde metodologias padrão de validação cruzada são aplicadas, descartando a importância \textit{temporal} na detecção de programas maliciosos, e também em condições de mundo real, onde a dependência temporal é proposta e aplicada.
Além disso, modificámos o nosso modelo original de modo a possibilitar a extração de mais informação sobre uma amostra, ao implementar um modelo com multiplas camadas, e de maneira a melhorar os resultados finais, ao usar informação dinâmica sobre a amostras.
Usámos depois a nossa metodologia baseada na ordem temporal das amostras para reduzir o tamanho dos dados de treino sem comprometer resultados ótimos, concluindo que existe um número ideal de dados de treino, temporalmente consistentes com os dados de validação, tal que o resultado final é ótimo.
Finalmente, fornecemos aplicações práticas do nosso modelo ao implementar o serviço de detecção de programas maliciosos, que é também inserido num servidor de correio eletrónico para validar anexos.

\vfill

\textbf{\Large Palavras-chave:} Segurança, Aprendizagem Automática, Detecção de Programas Maliciosos, Consistência Temporal, Verdade Absoluta

